\documentclass[draft, 12pt, sumlimits, intlimits]{article}

\usepackage[utf8]{inputenc}
\usepackage[T1]{fontenc}
\usepackage[english]{babel}
\usepackage{hyperref}
\usepackage{lmodern, microtype}
\usepackage[margin=4cm, paper=a4paper]{geometry}
\usepackage[margin=2cm]{caption}
\usepackage{booktabs}
\usepackage{graphicx}

\usepackage{amsfonts, amsmath, amssymb}
\usepackage{mathtools}
\newcommand \twopi{{{\scriptstyle(2}\mskip-2.0mu\pi{\scriptstyle)}}}
\newcommand \ee{{\mathrm e}}
\newcommand \ii{{\mathrm i}}
\newcommand \full{{\mathrm d}}
\newcommand \fulld[1]{{\frac \full{\full {#1}}}}
\newcommand \partiald[1]{{\frac \partial{\partial {#1}}}}
\newcommand \yesnumber{\addtocounter{equation}{1}\tag \theequation}

\usepackage{listings}
\lstset{basicstyle=\footnotesize\ttfamily, tabsize=2}

\usepackage[backend=bibtex]{biblatex}
\addbibresource{refs.bib}

\title{ECMI Modelling Week 2018 \\ Storing Your Random Objects}
\author{Desiré Nilsson \\ Lund University \\
\texttt{nat14dni@student.lu.se} \and
Ivana Gengeliacki \\ University of Novi Sad \\
\texttt{tim3ivana@gmail.com} \and
Jacob Hansen \\ Technical University of Denmark \\
\texttt{jacobjonhansen@gmail.com} \and
Kirill Kiselev \\ Saint Petersburg Polytechnic University \\
\texttt{kvladimirovich10@gmail.com} \and
Monika Žunji \\ University of Novi Sad \\
\texttt{zunjimonika@gmail.com} \and
Sampsa Kiiskinen \\ University of Jyväskylä \\
\texttt{tuplanolla@gmail.com}}
\date{2018-07-15 -- 2018-07-21}

\begin{document}

\maketitle

\section{Introduction}

Words that reference figure \ref{f/stuff} and
some literature \cite{conway-1998} go here.

\begin{figure}
  \centering
  $\wp$ % \includegraphics{stuff}
  \caption{Figure goes here.}
  \label{f/stuff}
\end{figure}

\section{Notes and Drafts}

We decided to start with a bottom-to-top reconstruction algorithm used
for granular dynamics \cite{poschel-2005},
but specialize it for rectangular particles.

\subsection{Bottom-to-Top Reconstruction}

What is the big picture (outline)?

\subsection{Line Intersection}

The intersection of two lines in a plane is easily obtained by solving
\begin{align*}
  X_1 + t_1 E_1 & = X_2 + t_2 E_2
  \yesnumber
\end{align*}
where $X$ are the positions (start vertices),
$E$ are the normalized direction vectors (vertex differences) and
$t$ are free.
If the lines are segments,
they intersect whenever $0 \le t_i \le 1$ for all $i$.
There also exists a three-dimensional generalization,
which yields the points,
where the two lines are the closest.

\subsection{Pivot Selection}

How to select the pivot?

\subsection{Rotation}

The rotation matrix around the origin
in the plane spanned by the first two dimensions is
\begin{align*}
  R (\theta) & =
  \begin{bmatrix}
    \cos \theta & -\sin \theta \\
    \sin \theta & \cos \theta
  \end{bmatrix}
  \yesnumber
\end{align*}
where $\theta$ is the angle.
In order to rotate around an arbitrary point,
we need to extend this to an affine transformation.
The familiar way is to use homogeneous coordinates.
We thus define the rotation matrix
\begin{align*}
  R' (\theta) & =
  \begin{bmatrix}
    \cos \theta & -\sin \theta & 0 \\
    \sin \theta & \cos \theta & 0 \\
    0 & 0 & 1
  \end{bmatrix}
  \yesnumber
\end{align*}
and the translation matrix
\begin{align*}
  T (x) & =
  \begin{bmatrix}
    1 & 0 & x_1 \\
    0 & 1 & x_2 \\
    0 & 0 & 1
  \end{bmatrix}.
  \yesnumber
\end{align*}
Composing them gives the matrix for a rotation around a point,
which we shall refer to as pivoting,
\begin{align*}
  P (x, \theta) & = T (x) R (\theta) T (-x).
  \yesnumber
\end{align*}

\subsection{Order Metrics}

In physics parlance,
quantitative ways to measure randomness
in a system are called order metrics.
There are various kinds \cite{torquato-2002}.
Alas, most assume the system consists of point particles,
features bonds (chemical or otherwise) or
tends to arrange into a known set of crystal structures.
Our system does not have these properties,
so we cannot employ them directly.

The best option we could lift from physics is
the two-body excess entropy \cite{truskett-2000},
which has the dimensionless form
\begin{align*}
  \frac S{k^{\mathrm B}} & = -\frac 1 2 \Big(\frac N V\Big)
  \int_0^\infty \fulld r \big(g \log g - (g - 1)\big),
  \label{e/entropy} \yesnumber
\end{align*}
where $N$ is the number of particles,
$V$ is the total volume of the ambient space,
$g$ is the radial distribution function parametrized
by the distance $r$ (obtainable
through weighted kernel density estimation \cite{kiiskinen-2018}) and
$k^{\mathrm B}$ is the Boltzmann constant.
However, this does not consider the orientation of the bodies.
To do that,
we could use the orientational correlation function \cite{donev-2006},
but it is not obvious how to incorporate it yet.

While the (unnormalized) radial correlation function says
\begin{align*}
  g_2 (r) & = \langle\delta (r - r')\rangle_{r'},
  \label{e/radial} \yesnumber
\end{align*}
the orientational correlation function
\begin{align*}
  g_m (r) & = \langle\cos m (\theta - \theta')\rangle_{\theta'},
  \label{e/angular} \yesnumber
\end{align*}
where $m$ is the degree of the symmetries (as in \cite{donev-2006}).
We conjecture that for squares $m = 4$ and for rectangles $m = 2$.
More is in \cite{saintillan-2007} and \cite{stoyan-1991}.
Unfortunately combining these is not discussed anywhere.

\printbibliography

\end{document}
